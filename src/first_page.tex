 \documentclass[12pt]{article}
\usepackage[utf8]{inputenc}
\usepackage{amsmath,amssymb,amsfonts} %need to use /dfrac

\title{ Tutorial on Latex}
\author{ dibrey jonatan}
\date{24 may 2025}

\begin{document}
\maketitle
\section{introduction}
This is my first latex document
$$ x^2$$
when we use \$ we are in math mode
To create an exponent in latex use dollar sign
$x^5$

NB : double dollar sign will put the equation in a new lign and will align 

To create subscripts we will have to make this 
$$x_1$$
$$x_{12}$$
$$x_{1_2}$$
$$x_{1_{2_3}}$$
\maketitle
\section{greek letters}
$$\pi$$
$$\Pi$$
$$\alpha$$
$$\beta$$
$$\gamma$$
\maketitle
\section{Trigonometric}
$$ y=\sin x $$
$$ y=\cos x $$
$$ y=\sec x $$
$$ y=\csc x $$
$$ y=\cot \alpha $$
$$ y=\tan \beta $$
\maketitle
\section{Inverse Trigonometric functions}
$$ y=\arcsin x $$
$$ y=\arccos x $$
$$ y=\arctan x$$
$$ y=\tan^{-1} x$$
\maketitle
\section{log functions}
$$y=\log x$$
$$ y=\log_5 x $$
$$ y=\ln x $$
\maketitle
\section{Roots} 
$$\sqrt{2}$$
To make cube roots or any other use this 
$$\sqrt[3]{2}$$

$$\sqrt{1+\sqrt{x}}$$
\maketitle
\section{Fractions}
$$\frac{1}{2}$$
$$\frac{1}{x^2+1}$$
To produce space from one line to another you will have to use this.\\[16pt]
So the distance is as you see.
You can make fraction to have same size as the text in it. watch the difference
between $\frac{1}{2}$ and $\dfrac{1}{2}$
\maketitle
\section{Brackets, arrays}
$${1,2,3}$$  %using \\ will bring the text to the next line
to print brackets do this $\{ 1,2,3 \}$ \\
To have bracket do this $\left(  s   \right)$ \\
To have square brackets $\left[ s \right]$ \\
To have curle bracket do this $\left\{ s  \right\}$ \\
To have angle do this $\left \langle  s \right\rangle $ \\
To have absolute value $\left |  s \right | $ \\
To print \$ sign : use the back slash \\
To put slash at one end $\left.  \right|_{x=1}$ \\

EXercice 
to produce the equation below:\\
$\left(\frac{1}{1+{\left(\frac {1}{1+x^2} \right)}} \right)$
\maketitle
\section{Tables}

\begin{tabular}{|c|c|c|c|} %precise the number of columns
    \hline
    $x$&1&2&3 \\ \hline
    $f(x)$&2&4&6 \\ \hline
    
\end{tabular}

\maketitle
\section{Arrays}
%when you use align they are automatically in math mode
\begin{align}
x^2+2x+1=(x+1)^2
\end{align}
\begin{align}
f(x)=sin(cos(x))
\end{align}


\maketitle
\section{Lists}
To have a an enumerated list
\begin{enumerate}
\item pen
\item calculator
\item ruler
\end{enumerate}

To have a dot list 
\begin{itemize}
\item pen
\item calculator
\item ruler
\end{itemize}
To have a nested list 
\begin{enumerate}
\item pen
\begin{enumerate}
\item red pen
\item blue pen
\item black pen
\end{enumerate}
\item calculator
\begin{itemize}
\item calcio
\item phone calc

\end{itemize}
\item ruler
\end{enumerate}
\begin{enumerate} %to remove the list numbering
\item[] pen
\item[] calculator
\item[] ruler
\end{enumerate}
\begin{enumerate} %to personalise 
\item[one] pen
\item[two] calculator
\item[three] ruler
\end{enumerate}
\end{document}